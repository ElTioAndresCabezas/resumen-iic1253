\documentclass[../main.tex]{subfiles}

\begin{document}

\section{Anexo}
\subsection{Ejercicio - Inferencia lógica de predicados}
\[
    \begin{tabular}{l}
        Algún estudiante en la sala no estudió para el examen.\\
        Todos los estudiantes de la sala pasaron el examen.\\ \hline
        Algún estudiante pasó el examen y no estudió.
    \end{tabular}
\]
¿Cómo modelamos este problema?
$$S(x) := x \text{ está en la sala}$$
$$E(x) := x \text{ estudió para el examen}$$
$$X(x) := x \text{ pasó el examen}$$
Entonces, la consecuencia lógica quedaría así:
\[
    \begin{tabular}{l}
        $\exists{} x . S(x) \wedge{} \neg{} E(x)$\\
        $\forall{} x . S(x) \rightarrow{} X(x)$\\ \hline
        $\exists{} x . X(x) \wedge{} \neg{} E(x)$
    \end{tabular}
\]
¿Cómo inferimos esta consecuencia lógica?
\begin{enumerate}
    \item $\exists x . S(x) \wedge \neg E(x)$ (Premisa)
    \item $S(a) \wedge \neg E(a)$ (Instanciación Existencial 1.)
    \item $S(a)$ (Simplificación Conjuntiva 2.)
    \item $\forall x . S(x) \rightarrow X(x)$ (Premisa)
    \item $S(a) \rightarrow X(a)$ (Instanciación Universal 4.)
    \item $X(a)$ (Modus ponens 3. y 5.)
    \item $\neg E(a)$ (Simplificación Conjuntiva 2.)
    \item $X(a) \wedge \neg E(a)$ (Conjunción 6. y 7.)
    \item $\exists x . X(x) \wedge \neg E(x)$ (Generalización Existencial 8.)
\end{enumerate}

\subsection{Demostración - Mínimo único}
\label{sec:dem_min_unico}
\textbf{Si $S$ tiene un elemento mínimo, entonces dicho elemento es único.}\\
Sea $x_1^{\downarrow} \in S$ y $x_2^{\downarrow} \in S$ ambos mínimos y $x_1 \not= x_2$. ¿Es esto posible?\\
Mediante la definición de mínimo, llegamos a que
\[ \forall y \in S . x_1^{\downarrow} \preceq y \quad ; \quad \forall y \in S . x_2^{\downarrow} \preceq y \]
Esto nos lleva a que
\[
    \left.
        \begin{array}{c}
            x_1^{\downarrow} \preceq x_2^{\downarrow}\\
            x_2^{\downarrow} \preceq x_1^{\downarrow}
        \end{array}
    \right \}
    x_1^{\downarrow} = x_2^{\downarrow}
\]
Así, queda demostrado que el mínimo debe ser único.

\subsection{Demostración - Si es mínimo, es minimal}
\label{sec:dem_min_minimal}
\textbf{Si $x$ es mínimo, entonces x es minimal}\\
Sea $x^{\downarrow}$ un mínimo\\
Para realizar la demostración, se debe demostrar que $\forall z \in S . z \preceq x^{\downarrow} \rightarrow z = x$\\
Suponga que $z \preceq x^{\downarrow}$. Por definición de mínimo, se tiene que $x^{\downarrow} \preceq z$. Entonces, considerando toda la información, se llega a que
\[
    \left.
        \begin{array}{c}
            z \preceq x^{\downarrow}\\
            x^{\downarrow} \preceq z
        \end{array}
    \right \}
    z = x^{\downarrow}
\]

\end{document}