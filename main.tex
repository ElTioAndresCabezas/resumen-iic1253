\documentclass{article}
\usepackage{amsmath}
\usepackage{fullpage}
\usepackage[table, x11names]{xcolor}
\usepackage[bookmarks]{hyperref}
\usepackage{dsfont}

\title{Resumen general de Matemáticas Discretas}
\author{Andrés Cabezas y Sebastián Poblete}

\begin{document}

\maketitle

\section{¿Qué son las matemáticas discretas?}
``El lenguaje necesario para entender y modelar la computación''. Las matemáticas discretas usan conjuntos finitos e infinitos al momento de estudio. Modelan los objetos y conceptos abstractos de las matemáticas que pueden ser representados dentro de un computador.
\subsection{Lógica}
La lógica consiste en el uso y estudio del razonamiento válido. Para esto, es necesario un lenguaje, pero esto también supone un problema: Los lenguajes que los humanos hablan tiene ciertas subjetividades y diferencias entre si, lo que conduce a errores al momento de usar la lógica. Para resolver esto, es necesario usar un lenguaje formal.\\
Durante el curso, se estudiarán dos lógicas (o lenguajes), sin embargo, existen muchos más
\begin{itemize}
    \item Lógica Proposicional
    \item Lógica de Predicados
\end{itemize}
\textbf{¿Para qué son necesarias estas lógicas?} Recordemos nuestro objetivo. Queremos usar esto para realizar nuestro razonamiento matemático. De esta forma, podemos definir correctamente objetos matemáticos, teorías matemáticas y realizar demostraciones más formales

\section{Lógica Proposicional (LP)}
\subsection{Proposición}
Una proposición consiste en una afirmación, la cual puede ser \textit{verdadera (1)} o \textit{falsa (0)}.\\
Para denotar proposiciones básicas, usaremos letras mayusculas (Ej.: P, Q, R\ldots)

\subsection{Conectivos Lógicos}
La LP usa conectivos sencillos para conseguir formar proposiciones más complejas. \\
\[
    \begin{tabular}{|c|c|c|c|}
        \hline
        Conectivos & Nombre & Uso & Significado\\ \hline
        $\wedge$ & Conjunción & $P \wedge{}Q$ & P y Q \\
        $\vee$ & Disyunción & $P \vee{}Q$ & P o Q \\
        $\neg$ & Negación & $\neg{}P$ & No P \\
        $\rightarrow$ & Condicional & $P \rightarrow{}Q$ & Si P, entonces Q \\
        $\leftrightarrow$ & Bicondicional & $P \leftrightarrow{}Q$ & P, si y solo si, Q \\ \hline
    \end{tabular}
\]
\subsubsection{Conjunción ($\wedge$)}
El valor de verdad de una conjunción es \textit{verdadero} si ambas proposiciones (a cada lado del signo) son verdaderas. En cualquier otro caso, es \textit{falso}.
\[
    \begin{tabular}{cc|c}
        P & Q & $P \wedge{}Q$\\ \hline
        0 & 0 & 0\\
        0 & 1 & 0\\
        1 & 0 & 0\\
        1 & 1 & 1
    \end{tabular}
\]

\subsubsection{Disyunción ($\vee$)}
El valor de verdad de una disyunción es \textit{verdadero} si al menos una de las proposiciones (a cada lado del signo), es verdadera.
\[
    \begin{tabular}{cc|c}
        P & Q & $P \vee{}Q$\\ \hline
        0 & 0 & 0\\
        0 & 1 & 1\\
        1 & 0 & 1\\
        1 & 1 & 1
    \end{tabular}
\]

\subsubsection{Negación ($\neg$)}
El valor de verdad corresponde al opuesto del valor entregado (a la derecha del signo)
\[
    \begin{tabular}{c|c}
        P & $\neg{}P$\\ \hline
        1 & 0\\
        0 & 1 
    \end{tabular}
\]

\subsubsection{Condicional ($\rightarrow$)}
El valor de verdad de una condicional del tipo $P \rightarrow{}Q$ es \textit{falso} si P es verdadero, pero Q es falso. En cualquier otro caso, es \textit{verdadero}.\\
\[
    \begin{tabular}{cc|c}
        P & Q & $P \rightarrow{}Q$\\ \hline
        0 & 0 & 1\\
        0 & 1 & 1\\
        1 & 0 & 0\\
        1 & 1 & 1
    \end{tabular}
\]
Hint: \textit{``Si P es verdadero, entonces necesariamente Q es verdadero''}. Si P es verdadero, entonces Q deberá ser verdadero para tener un valor de verdad \textit{verdadero}. Si P es falso, entonces de forma automática el valor de verdad es \textit{verdadero}.

\subsubsection{Bicondicional ($\leftrightarrow$)}
El valor de verdad de una bicondicional es verdadero si ambas proposiciones (a ambos lados del signo) son iguales (en otras palabras, ambas verdaderas o ambas falsas).
\[
    \begin{tabular}{cc|c}
        P & Q & $P \leftrightarrow{}Q$\\ \hline
        0 & 0 & 1\\
        0 & 1 & 0\\
        1 & 0 & 0\\
        1 & 1 & 1
    \end{tabular}
\]

\subsection{Proposición Compuesta}
Una proposición es compuesta si corresponde a la negación ($\neg$), conjunción ($\wedge$), disyunción($\vee$), condicional($\rightarrow$) o bicondicional( $\leftrightarrow$ ) de proposiciones compuestas.\\
Como por ejemplo


$$P \wedge (Q \vee{} R)$$
$$\neg(P \vee (\neg{}R \wedge{} Q))$$
$$(P \rightarrow{} Q) \leftrightarrow{} (P \wedge{} Q)$$



Si se desea obtener el valor de verdad de alguna proposición compuesta, se debe evaluar de forma recursiva cada uno de los conectivos lógicos presentes.\\
Por ejemplo:

$$\neg(P \vee (\neg{} R \wedge{} Q)) \textrm{ con P = 0, Q = 1 y R = 0}$$
$$\neg(0 \vee (\neg{} 0 \wedge{} 1))$$
$$\neg(0 \vee (1 \wedge{} 1))$$
$$\neg(0 \vee 1)$$
$$\neg{}1$$
$$0$$
\\
$$(P \rightarrow{} Q) \leftrightarrow{} (P \wedge{} Q) \text{ con P = 1 y Q = 0}$$
$$(1 \rightarrow{} 0) \leftrightarrow{} (1 \wedge{} 0)$$
$$0 \leftrightarrow{} 0$$
$$1$$

\subsubsection{Paréntesis y prioridad}
El orden de prioridad entre conectivos lógicos, al momento de evaluar proposiciones compuestas, será el siguiente:
\[
    \begin{tabular}{c|c}
        Conectivo & Precedencia\\ \hline
        $\neg$ & 1\\
        $\wedge$ & 2\\
        $\vee$ & 3\\
        $\rightarrow$ & 4\\
        $\leftrightarrow$ & 5
    \end{tabular}
\]

\section{Formulas y Valuaciones}
\subsection{Variables Proposicionales}
Una variable proposicional es una variable que puede ser reemplazada con los valores 1 o 0. Generalmente son representadas con una letra minúscula (Amiga eri boolean)

\subsection{Formulas Proposicionales}
Una formula proposicional es una formula que puede ser
\begin{itemize}
    \item Una variable proposicional
    \item Los valores 1 o 0
    \item Una combinación con conectivos lógicos
\end{itemize}
Generalmente son representadas con letras griegas (Ej.: $\alpha$)\\
Ejemplos:
$$\alpha{}(p,q,r) := p \wedge{}(q \rightarrow{} r)$$
$$\beta{}(p,q) := (p \wedge{} \neg{}q) \vee{} (\neg{}p \wedge{} 1)$$

\section{Equivalencia Lógica}
\label{sec:equiv_logica}
\subsection{Definición}
Si tenemos dos formulas proposicionales con las mismas variables proposicionales
$$\alpha{}(p_{1},\ldots{},p_{n}) \text{ y } \beta{}(p_{1},\ldots{},p_{n})$$
Entonces, $\alpha$ y $\beta$ serán logicamente equivalentes
$$\alpha \equiv \beta$$
si para toda valuación posible $(v_{1}, \ldots{},v_{n})$ se cumple que:
$$\alpha{}(v_{1}, \ldots{},v_{n}) = \beta{}(v_{1}, \ldots{},v_{n})$$
Ejemplo:
Para las fórmulas $p \wedge (q \vee r)$ y $(p \wedge q) \vee (p \wedge r)$ se tiene la siguiente tabla de verdad:
\[
    \begin{tabular}{ccc|cc}
        p & q & r & $p \wedge (q \vee r)$ & $(p \wedge q) \vee (p \wedge r)$ \\ \hline
        0 & 0 & 0 & 0 & 0\\
        0 & 0 & 1 & 0 & 0\\
        0 & 1 & 0 & 0 & 0\\
        0 & 1 & 1 & 0 & 0\\
        1 & 0 & 0 & 0 & 0\\
        1 & 0 & 1 & 1 & 1\\
        1 & 1 & 0 & 1 & 1\\
        1 & 1 & 1 & 1 & 1
    \end{tabular}
\]
Como ambas formulas son equivalentes para toda valuación, entonces:
$$p \wedge (q \vee r) \equiv (p \wedge q) \wedge (p \wedge r)$$

\subsection{Equivalencias útiles}
\label{sec:equiv_logica_util}
\begin{enumerate}
    \item Conmutatividad: 
                        $$p \wedge q \equiv q \wedge p$$ 
                        $$p \vee q \equiv q \vee p$$
    \item Asociatividad: 
                        $$p \wedge (q \wedge r) \equiv (p \wedge q) \wedge r$$
                        $$p \vee (q \vee r) \equiv (p \vee q) \vee r$$
    \item Idempotente:
                        $$p \wedge p \equiv p$$
                        $$p \vee p \equiv p$$
    \item Doble negación:
                        $$\neg\neg p \equiv p$$
    \item Distributividad:
                        $$p \wedge (q \vee r) \equiv (p \wedge q) \vee (p \wedge r)$$
                        $$p \vee (q \wedge r) \equiv (p \vee q) \wedge (p \vee r)$$
    \item De Morgan:
                        $$\neg (p \wedge q) \equiv \neg p \vee \neg q$$
    \item Implicación:
                        $$p \rightarrow q \equiv \neg p \vee q$$
                        $$p \rightarrow q \equiv \neg q \rightarrow \neg p$$
                        $$p \leftrightarrow q \equiv (p \rightarrow q) \wedge (q \rightarrow p)$$
    \item Absorción:
                        $$p \vee (p \wedge q) \equiv p$$
                        $$p \wedge (p \vee q) \equiv p$$
    \item Identidad:
                        $$p \vee 0 \equiv p$$
                        $$p \wedge 1 \equiv p$$
    \item Dominación:
                        $$p \wedge 0 \equiv 0$$
                        $$p \vee 1 \equiv 1$$
\end{enumerate}

\section{Operadores Generalizados}
Debido a que $\vee$ y $\wedge$ son operadores asociativos, podemos escribir las siguientes generalizaciones
$$\bigvee_{i=1}^{n} p_{i} \equiv{} p_{1} \vee{} p_{2} \vee{} \ldots{} \vee{} p_{n}$$
$$\bigwedge_{i=1}^{n} p_{i} \equiv{} p_{1} \wedge{} p_{2} \wedge{} \ldots{} \wedge{} p_{n}$$
Además, podemos saltarnos los parentesis al momento de escribir estas operaciones. Por ejemplo:
$$(p_{1} \vee{} p_{2}) \vee{} p_{3} \equiv{} p_{1} \vee{} (p_{2} \vee{} p_{3}) \equiv{} p_{1} \vee{} p_{2} \vee{} p_{3}$$
$$(p_{1} \wedge{} p_{2}) \wedge{} p_{3} \equiv{} p_{1} \wedge{} (p_{2} \wedge{} p_{3}) \equiv{} p_{1} \wedge{} p_{2} \wedge{} p_{3}$$


\section{Formas normales}
Primero, consideremos que un \textit{literal} es una variable proposicional o la negación de una variable.\\
\subsection{Forma Normal Disyuntiva (DNF)}
Una formula $\alpha$ está en DNF si es una disyunción de conjunciones de literales.
$$\alpha{} = \beta{}_{1} \vee{} \beta{}_{2} \vee{} \ldots{} \vee{} \beta{}_{k}$$
donde $\beta{}_{i} = (l_{i_{1}} \wedge{} \ldots{} \wedge{} l_{i_{k_{i}}})$ y $l_{i_{1}}, \ldots{}, l_{i_{k_{i}}}$ son literales.\\
Por si esta forma de anotarlo es muy complicada de entender, en palabras más simples, nos referimos a que $\alpha$ está en DNF si es que es una formula proposicional tal que este compuesta por disyunciones de otras formulas, las cuales son conjunciones de variables proposicionales.\\
Ejemplo:
    $$(p \wedge{} \neg{} q) \vee{} (\neg{} p \wedge{} p \wedge{} s) \vee{} (r \wedge{} \neg{} s)$$

\subsection{Forma Normal Conjuntiva (CNF)}
Una formula $\alpha$ está en CNF si es una conjuncion de disyunciones de literales.
$$\alpha{} = \beta{}_{1} \wedge{} \beta{}_{2} \wedge{} \ldots{} \wedge{} \beta{}_{k}$$
donde $\beta{}_{i} = (l_{i_{1}} \vee{} \ldots{} \vee{} l_{i_{k_{i}}})$ y $l_{i_{1}}, \ldots{}, l_{i_{k_{i}}}$ son literales.\\
Al final, la idea de la CNF es algo así como una ``forma inversa'' de la DNF, ya que se invierte que es lo que se encuentra en disyunción y lo que se encuentra en conjunción.\\
Ejemplo:
    $$(p \vee{} \neg{} q) \wedge{} (\neg{} p \vee{} p \vee{} s) \wedge{} (r \vee{} \neg{} s)$$

\subsection{Formas normales y Equivalencia lógica}
Tenemos el siguiente teorema:
\begin{enumerate}
    \item Toda formula $\alpha$ es lógicamente equivalente a una formula en DNF.
    \item Toda formula $\alpha$ es lógicamente equivalente a una formula en CNF.
\end{enumerate}
Debido a las limitadas herramientas de demostración que tenemos por el momento, la demostración será escrita a futuro en esta parte del resumen. En caso de estar estudiando con este resumen, no considere la demostración hasta tener las herramientas suficientes como para demostrar.

\section{Consecuencia Lógica}
Sea $\Sigma{} = \{\alpha{}_{1}, \ldots{}, \alpha{}_{m} \}$ un conjunto de formulas con variables $p_{1}, \ldots{}, p_{n}$. Diremos que $\alpha$ es \textit{consecuencia lógica} de $\Sigma$ si, y solo si, para toda valuación $v_{1}, \ldots{}, v_{n}$ se tiene que
\[ \text{si } [ \bigwedge_{i=1}^{m} \alpha{}_{i} ] (v_{1}, \ldots{}, v_{n}) = 1 \text{, entonces } \alpha{}(v_{1}, \ldots{}, v_{n}) = 1 \]
Esto se denota como $\Sigma{} \models{} \alpha$ (leido como $\alpha$ \textit{es consecuencia lógica de} $\Sigma$)\\
Posiblemente no quede del todo claro el significado de esta formula, pero en palabras, lo que queremos decir es que si tenemos una valuación, tal que al aplicarla a toda formula presente en el conjunto retorne 1, entonces si una formula es consecuencia lógica del conjunto, esta debe también retornar 1.\\
Si lo intentamos ver con una tabla de verdad, a modo de ejemplo, una consecuencia lógica se vería de la siguiente 
forma:
\[
    \begin{tabular}{ccc|cccc|c}
        $v_{1}$ & $\cdots{}$ & $v_{n}$ & $\alpha{}_{1}$ & $\alpha{}_{2}$ & $\cdots{}$ & $\alpha{}_{m}$ & $\alpha$ \\ \hline
                             $\cdots{}$ & $\cdots{}$ & $\cdots{}$ & 1 & 1 & $\cdots$ & 0 & 1 \\
        \rowcolor{lightgray} $\cdots{}$ & $\cdots{}$ & $\cdots{}$ & 1 & 1 & $\cdots$ & 1 & 1 \\
                             $\cdots{}$ & $\cdots{}$ & $\cdots{}$ & 0 & 0 & $\cdots$ & 1 & 0 \\
    \end{tabular}
\]
En donde tenemos una fila marcada en gris, en donde podemos ver que todas las formulas del conjunto $\Sigma$ son iguales a 1 y tambien que $\alpha$ es 1. Si esto se cumple y no sucede que tenemos todas las formulas de la izquierda con unos, y el alpha de la derecha con un 0, entonces tenemos consecuencia lógica.\\
Otro ejemplo:
\[
    \begin{tabular}{ccc|cccc|c}
        $v_{1}$ & $\cdots{}$ & $v_{n}$ & $\alpha{}_{1}$ & $\alpha{}_{2}$ & $\cdots{}$ & $\alpha{}_{m}$ & $\alpha$ \\ \hline
                             $\cdots{}$ & $\cdots{}$ & $\cdots{}$ & 1 & 1 & $\cdots$ & 0 & 1 \\
        \rowcolor{lightgray} $\cdots{}$ & $\cdots{}$ & $\cdots{}$ & 1 & 1 & $\cdots$ & 1 & 0 \\
                             $\cdots{}$ & $\cdots{}$ & $\cdots{}$ & 0 & 0 & $\cdots$ & 1 & 0 \\
    \end{tabular}
\]
En este caso, ya no hay consecuencia lógica, debido a que no se cumple la condición establecida antes. La fila marcada en gris, tiene un 0 en la formula final, lo que nos dice que no es consecuencia lógica.\\
\textbf{NOTA:} Si tenemos un conjunto $\Sigma$ tal que es imposible obtener una fila con solo unos, entonces \textbf{cualquier cosa} puede ser consecuencia lógica de $\Sigma$. ¡Usar con sabiduria para demostraciones!

\subsection{Consecuencias lógicas clásicas}
\subsubsection{Modus ponens}
$$ \{ p, p \rightarrow{} q \} \models{} q $$
\[
    \begin{tabular}{cc|cc|c}
        p & q & $p$ & $p \rightarrow{} q$ & $q$ \\ \hline
        0 & 0 & 0 & 1 & 0 \\
        0 & 1 & 0 & 1 & 0 \\
        1 & 0 & 1 & 0 & 0 \\
        \rowcolor{lightgray} 1 & 1 & 1 & 1 & 1 \\
    \end{tabular}
\]

\subsubsection{Modus tollens}
$$\{ \neg{} q, p \rightarrow q \} \models{} \neg{} p$$
\[
    \begin{tabular}{cc|cc|c}
        p & q & $p$ & $p \rightarrow{} q$ & $q$ \\ \hline
        \rowcolor{lightgray} 0 & 0 & 1 & 1 & 1 \\
        0 & 1 & 0 & 1 & 1 \\
        1 & 0 & 1 & 0 & 0 \\
        1 & 1 & 0 & 1 & 0 \\
    \end{tabular}
\]

\subsubsection{Resolución}
$$\{ p \vee{} q, \neg{} q \vee{} r \} \models{} p \vee{} r$$
\[
    \begin{tabular}{ccc|cc|c}
        p & q & r & $p \vee{} q$ & $\neg{} q \vee{} r$ & $p \vee{} r$ \\ \hline
        0 & 0 & 0 & 0 & 1 & 0\\
        0 & 0 & 1 & 0 & 1 & 1\\
        0 & 1 & 0 & 1 & 0 & 0\\
        \rowcolor{lightgray} 0 & 1 & 1 & 1 & 1 & 1\\
        \rowcolor{lightgray} 1 & 0 & 0 & 1 & 1 & 1\\
        \rowcolor{lightgray} 1 & 0 & 1 & 1 & 1 & 1\\
        1 & 1 & 0 & 1 & 0 & 1\\
        \rowcolor{lightgray} 1 & 1 & 1 & 1 & 1 & 1\\
    \end{tabular}
\]

\subsection{Trucos de consecuencia lógica}
\begin{enumerate}
    \item $\{ 1 \} \models{} \alpha{}$, entonces $\alpha$ es una tautología
    \item Si $\alpha$ es una contradicción, entonces $\{ \alpha{} \} \models{} \beta{}$
    \item Si $\Sigma{} \models{} \alpha{}$, entonces $\Sigma{} \cup{} \{ \beta{} \} \models{} \alpha{}$ para todo $\beta$
    \item Si $\Sigma{} \cup{} \{ \alpha{} \} \models{} \beta{}$ y $\Sigma{} \models{} \alpha{}$, entonces $\Sigma{} \models{} \beta{}$
\end{enumerate}

\subsection{Más consecuencias lógicas clásicas}
A continuación, vamos a mostrar más consecuencias lógicas, sumando a la lista en 7.1
\begin{enumerate}
    \item Modus ponens: $ \{ p, p \rightarrow{} q \} \models{} q $
    \item Modus tollens: $\{ p \vee{} q, \neg{} q \vee{} r \} \models{} p \vee{} r$
    \item Silogismo: $\{ p \rightarrow{} q, q \rightarrow{} r \} \models{} p \rightarrow{} r$
    \item Silogismo disyuntivo: $\{ p \vee{} q, \neg{} p \} \models{} q$
    \item Conjunción: $\{ p, q \} \models{} p \wedge{} q$
    \item Simplificación Conjuntiva: $\{ p \wedge{} q \} \models p$
    \item Amplificación Disyuntiva: $ \{ p \} \models{} p \vee{} q$
    \item Demostración Condicional: $\{ p \wedge{} q, p \rightarrow{} (q \rightarrow{} r) \} \models{} r$
    \item Demostración por casos: $\{ p \rightarrow{} r, q \rightarrow{} r \} \models (p \vee{} q) \rightarrow{} r$
\end{enumerate}

\subsection{Composición y Consecuencia lógica}
\subsubsection{Definición}
Considerando un conjunto $\Sigma{} = \{ \alpha{}_{1}(p_{1}, \ldots{}, p_{n}), \ldots{}, \alpha{}_{m}(p_{1}, \ldots{}, p_{n}) \}$ y $\beta{}_{1}, \ldots{}, \beta{}_{n}$ como formulas proposicionales.\\
Una composición $\Sigma{}(\beta{}_{1}, \ldots{}, \beta{}_{n})$, consiste en el conjunto resultante de valuar cada formula de $\Sigma$ con $\beta{}_{1}, \ldots{}, \beta{}_{n}$. En otras palabras:\\
\large $$\Sigma{}(\beta{}_{1}, \ldots{}, \beta{}_{n}) = \{ \alpha{}_{1}(\beta{}_{1}, \ldots{}, \beta{}_{n}), \ldots{}, \alpha{}_{m}(\beta{}_{1}, \ldots{}, \beta{}_{n}) \}$$
\normalsize

\subsubsection{Teorema}
Sea $\Sigma$ un conjunto de formulas (similar al de antes), y $\alpha{}, \beta{}_{1}, \ldots{}, \beta{}_{n}$, formulas proposicionales. Si $\Sigma{} \models{} \alpha{}$, entonces $\Sigma{}(\beta{}_{1}, \ldots{}, \beta{}_{n}) \models{} \alpha{}(\beta{}_{1}, \ldots{}, \beta{}_{n})$.

\section{Satisfacibilidad}
\subsection{Satisfacción de un conjunto de formulas}
Se dice que una formula proposicional $\alpha(p_{1}, \ldots{}, p_{n})$ es satisfacible si existe una valuación $v_{1}, \ldots{}, v_{n}$ tal que
$$\alpha{}(v_{1}, \ldots{}, v_{n}) = 1$$
Un conjunto $\Sigma = \{ \alpha{}_{1}, \ldots{}, \alpha{}_{m} \}$ con variables $p_{1}, \ldots{}, p_{n}$ se dice que es satisfacible si existe una valuación $v_{1}, \ldots{}, v_{n}$ tal que
$$[ \bigwedge_{i = 1}^{m} = \alpha{}_{i} ] (v_{1}, \ldots{}, v_{n}) = 1$$
Si un conjunto o formula no es satisfacible, entonces se dice que es inconsistente.\\

\subsection{Consecuencia lógica vs satisfacibilidad}
\textbf{Teorema:} $\{\alpha{}_{1}, \ldots{}, \alpha{}_{m}\} \models{} \alpha{}$ si y solo si $\{ \alpha{}_{1}, \ldots{}, \alpha{}_{m}, \neg{} \alpha \}$ es inconsistente.

\subsection{Satisfacibilidad y representación de problemas}
\textbf{Problema:} Dada una formula $\alpha$, verificar si es o no satisfacible.\\
¿Como podemos resolver este problema? Si bien es posible ir probando todas las posibles valuaciones para verificar, es un proceso largo y poco eficiente. Tristemente, la respuesta a este problema, es que no es posible. No existe ningun otro método al momento de verificar si una formula es o no es consistente.

\section{Lógica de Predicados}
Hasta ahora, se ha trabajado unicamente con Lógica Proposicional. Esta funciona bien, pero tiene algunas limitaciones que vuelven imposible modelar algunas situaciones.
\begin{itemize}
    \item No tiene objetos. Solo se pueden usar proposiciones.
    \item No tiene predicados.
    \item No tiene cuantificadores.
\end{itemize}

\subsection{¿Y qué tiene la Lógica de Predicados?}
La lógica de predicados es una parte de la lógica de primer orden. La lógica de predicados nos va a permitir expresar ciertas estructuras las cuales no eramos capaces de expresar usando la lógica proposicional.

\subsection{Predicados}
Un predicado consiste en una proposicion abierta. El valor de verdad de un predicado dependerá del valor usado en la valuación. Generalmente, estos se simbolizan usando letras mayúsculas (Ej.: $P(x)$)\\
Ejemplos de predicados:
\begin{itemize}
    \item $P(x)$ := x es par
    \item $R(x)$ := x es primo
    \item $M(x)$ := x es mortal
\end{itemize}

\subsubsection{Predicados n-arios}
Los predicados n-arios consisten en predicados los cuales usan más de una variable para verificar su valor de verdad.\\
Ejemplo: $O(x,y) := x \leq y $. Si $x = 2$ e $y = 3$, entonces $O(2,3) = 1$

\subsubsection{Dominio de predicado}
Todo predicado está restringido a un cierto dominio de evaluación. Esto significa que sus valores de verdad solo se pueden evaluar cuando las variables que se usan en el predicado estan dentro del dominio designado.
Ej.: $O(x,y) := x \leq{} y$ sobre $\mathds{N}$

\subsubsection{Predicado 0-ario / Predicado degenerado}
Corresponde a un predicado que no tiene ninguna variable libre. Tiene un valor de verdad el cual es totalmente independiente de su valuación.

\subsubsection{Predicados compuestos}
Un predicado compuesto corresponde a la combinación de diversos predicados básicos, usando distintos operadores para mezclarlos en la sentencia, o la cuantificación universal o existencial de algun predicado (Véase \hyperref[sec:cuantificadores]{sección 9.3}). Todos los predicados deben tener el mismo dominio.

\subsection{Cuantificadores}
\label{sec:cuantificadores}
\subsubsection{Cuantificador Universal}
Consideremos $P(x, y_{1}, \ldots{}, y_{n})$ un predicado compuesto de dominio \textit{D}.\\
El cuantificador universal corresponde a
$$P'(y_{1}, \ldots{}, y_{n}) := \forall{} x . P(x, y_{1}, \ldots{}, y_{n})$$
donde $x$ es la variable cuantificada e $y_{1}, \ldots{}, y_{n}$ son las variables libres.\\
Si para cierta valuación se cumple que $P'(\ldots{}) = 1$ significa que para toda variable libre se cumple lo especificado anteriormente.\\
\textbf{Ejemplo}
$$O(x,y) := x \leq y \text{ sobre } \mathds{N}$$
$$O'(y) := \forall x . O(x,y)$$
$$O'(2) = \forall x. O(x,2) = 0$$
Podemos notar que para $y = 2$, no se cumple para todo valor de $x$ lo dictado en el predicado $O(x,y)$
$$O''(x) := \forall y . O(x,y)$$
$$O''(0) := \forall y . O(0,y) = 1$$
Si $x = 0$, debido a que estamos considerando a los naturales, entonces para todo valor de $y$ se cumple siempre el predicado.

\subsubsection{Cuantificador Existencial}
Consideremos $P(x, y_{1}, \ldots{}, y_{n})$ un predicado compuesto de dominio \textit{D}.\\
El cuantificador existencial corresponde a
$$P'(y_{1}, \ldots{}, y_{n}) := \exists x . P(x, y_{1}, \ldots{}, y_{n})$$
donde $x$ es la variable cuantificada e $y_{1}, \ldots{}, y_{n}$ son las variables libres.\\
Si para cierta valuación se cumple que $P'(\ldots{}) = 1$ significa que para alguna combinación de variables libres se cumple lo especificado anteriormente.\\
\textbf{Ejemplo}
$$O(x,y) := x \leq y \text{ sobre } \mathds{N}$$
$$O'(y) := \exists x . O(x,y)$$
$$O'(2) = \exists x. O(x,2) = 1$$

\subsection{Interpretaciones}
En algunos casos, puede ocurrir que algún predicado o formula, dependiendo de ciertas condiciones, sea verdadero(a)
o falso(a). Debido a esto, vamos a definir las \textit{interpretaciones}.\\
Para comenzar, es importante destacar que desde ahora diremos que $P(x_{1}, \ldots, x_{n})$ es un símbolo de predicado.\\
Una interpretación $\mathcal{I}$ para símbolo de predicado $P_{1}, \ldots, P_{m}$ se compone de
\begin{itemize}
    \item Un dominio $\mathcal{I}(\text{dom})$
    \item Para cada símbolo $P_{i}$, un predicado $\mathcal{I}(P_{i})$
\end{itemize}
Sea $\alpha{}(x_{1}, \ldots, x_{n})$ una formula y $\mathcal{I}$ una interpretación de los símbolos en $\alpha$. Se dice que la interpretación $\mathcal{I}$ satisface $\alpha$ sobre $a_{1}, \ldots, a_{n}$ en $\mathcal{I}(\text{dom})$, expresado como
$$\mathcal{I} \models \alpha{}(a_{1}, \ldots, a_{n})$$
si $\alpha{}(a_{1}, \ldots, a_{n})$ es verdadero al evaluar cada símbolo en $\alpha$ según $\mathcal{I}$. En el caso que $\mathcal{I}$ no logre satisfacer a $\alpha$, se anotará como
$$\mathcal{I} \not\models \alpha{}(a_{1}, \ldots, a_{n})$$
\textbf{Ejemplo}
$$\mathcal{I}_{1}(\text{dom}) := \mathds{N}$$
$$\mathcal{I}_{1}(P) := \text{x es par}$$
$$\mathcal{I}_{1}(O) := x < y$$
$$\alpha{}(x) := \exists y . P(y) \wedge O(x,y)$$
$$\mathcal{I}_{1} \models \alpha{}(1) := \exists y. \text{y es par} \wedge 1 < y$$

\section{Equivalencia lógica en Lógica de Predicados}
Se tiene $\alpha{}(x_{1}, \ldots, x_{n})$ y $\beta{}(x_{1}, \ldots, x_{n})$ dos oraciones en lógica de predicados (no tienen variables libres). $\alpha$ y $\beta$ serán logicamente equivalentes, escrito como:
$$\alpha \equiv \beta$$
si para toda interpretación $\mathcal{I}$ y para todo $a_{1}, \ldots, a_{n}$ se cumple que:
$$\mathcal{I} \models \alpha{}(a_{1}, \ldots, a_{n}) \text{ si, y solo si, } \mathcal{I} \models \beta{}(a_{1}, \ldots, a_{n})$$
En otras palabras, funciona practicamente igual a la \hyperref[sec:equiv_logica]{equivalencia lógica en lógica 
proposicional}.

\subsection{Equivalencias lógicas}
Todas las \hyperref[sec:equiv_logica_util]{equivalencias de lógica proposicional} aplican aquí. Simplemente se tienen que cambiar las variables proposicionales por predicados de lógica de predicados. Aquí se muestra un ejemplo:\\
\textbf{Conmutatividad} (para operador $\wedge$)
\[    
    \begin{tabular}{|c|c|}
        \hline
        En lógica proposicional & En lógica de predicados\\ \hline
        $p \wedge{} q \equiv{} q \wedge{} p$ & $\alpha{} \wedge{} \beta{} \equiv{} \beta{} \wedge{} \alpha{}$ \\ \hline
    \end{tabular}
\]
Ahora, además de estas equivalencias, nos encontraremos con algunas nuevas
\begin{enumerate}
    \item $\neg \forall x . \alpha \equiv \exists x . \neg \alpha$
    \item $\neg \exists x . \alpha \equiv \forall x . \neg \alpha$
    \item $\forall x . (\alpha \wedge \beta) \equiv (\forall x . \alpha) \wedge (\forall x . \beta)$
    \item $\forall x . (\alpha \vee \beta) \equiv (\forall x . \alpha) \vee (\forall x . \beta)$
\end{enumerate}

\end{document}