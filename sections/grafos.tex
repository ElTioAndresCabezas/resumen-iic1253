\documentclass[../main.tex]{subfiles}

\begin{document}

\section{Grafos}
Un grafo no dirigido $G$ es un par $(V,E)$ donde $V \not= \emptyset$ corresponde al conjunto de nodos (o vertices) y $E \subseteq 2^V$ corresponde al conjunto de conexiones o aristas del grafo (destacar que todo $e \in E$ debe ser una tupla de dos valores, como por ejemplo $e = \{a,b\}$).

Ejemplo:
\[
    \left.
    \begin{tikzpicture}
        \node (a) {a};
        \node (b) [right=of a] {b};
        \node (c) [below=of a] {c};
        \node (d) [right=of c] {d};
        \node (e) [right=of b] {e};
        \node (f) [below=of e] {f};

        \draw (a) -- (b);
        \draw (a) -- (d);
        \draw (a) -- (c);
        \draw (b) -- (e);
        \draw (c) -- (d);
        \draw (d) -- (e);
        \draw (d) -- (f);
        \draw (e) -- (f);
    \end{tikzpicture}
    \right. \hspace{1em}\raisebox{3em}{ \begin{tabular}{rcl}
        $V$ & $=$ & $\{ a,b,c,d,e,f \}$\\
        $E$ & $=$ & $\{ \{ a,b \}, \{ a,c \}, \{ a,d \}, \{ b,e \}, \{ c,d \}, \{ d,e \}, \{ d,f \}, \{ e,f \} \}$
    \end{tabular}
    }
\]

Cabe destacar que en un grafo no dirigido como este, no hay loops (conexiones que vayan de un nodo al mismo nodo), las aristas no tiene dirección (lo que significa que $\{ a,b \} = \{ b,a \}$) y para cada par de nodos solo puede haber una conexión.

\subsection{Familias de grafos}
\subsubsection{Grafos Completos (Cliques)}
Un grafo completo o clique corresponde a un grafo de $n$ vertices tal que $K_n = (V,E)$, $|V| = n$ y $E = \{ \{u,v\} | u,v \in V \wedge u \not= v \}$. De forma más sencilla, un grafo completo es un grafo que conecta todos los nodos que contiene con todos los demás nodos.
\[
    \begin{tikzpicture}
        \node (1) {1};
        \node (empty_top) [left=of 1] {};
        \node (2) [left=of empty_top] {2};
        \node (empty1) [below=of 2] {};
        \node (3) [left=of empty1] {3};
        \node (empty2) [below=of 1] {};
        \node (5) [right=of empty2] {5};
        \node (empty3) [below=of empty_top] {};
        \node (4) [below=of empty3] {4};

        \draw (1) -- (2);
        \draw (1) -- (3);
        \draw (1) -- (4);
        \draw (1) -- (5);

        \draw (2) -- (3);
        \draw (2) -- (4);
        \draw (2) -- (5);

        \draw (3) -- (4);
        \draw (3) -- (5);

        \draw (4) -- (5);
        
    \end{tikzpicture}
\]

\subsubsection{Linea}
Un grafo de tipo linea consiste en un grafo de $n$ vertices tal que $L_n = (V,E)$, $V = \{ 1, 2, 3, \ldots, n - 1 \}$ y $E = \{ \{ i, i + 1 \} | 0 \leq i \leq n - 1 \}$. En palabras simples, la linea corresponde a un grafo en donde todos los nodos se conectan con el ``punto siguiente'' (asumiendo que se tiene un orden establecido previamente, como $1$, $2$, $3$, etc.)
\[
    \begin{tikzpicture}
        \node (1) {1};
        \node (2) [right=of 1] {2};
        \node (3) [right=of 2] {3};
        \node (4) [right=of 3] {4};
        \node (5) [right=of 4] {5};

        \draw (1) -- (2);
        \draw (2) -- (3);
        \draw (3) -- (4);
        \draw (4) -- (5);
    \end{tikzpicture}
\]

\subsubsection{Ciclo}
Un grafo cíclico corresponde a un grafo de $n$ vertices tal que $C_n = (V,E)$, $V = \{ 0, 1, \ldots, n - 1 \}$ y $E = \{ \{ i, (i + 1) \mod n \} | 0 \leq i \leq n - 1 \}$. En otras palabras, corresponde a un grafo el cual forma una figura cíclica, en donde el último elemento de la cadena se debe unir con el primero.

\[
    \begin{tikzpicture}
        \def \n {5}
        \def \radius {3cm}
        \def \margin {8} % margin in angles, depends on the radius

        \foreach \s in {1,...,\n}{
            \node (\s) at ({360/\n * (\s - 1)}:\radius) {$\s$};
        }
        \draw (1) -- (2);
        \draw (2) -- (3);
        \draw (3) -- (4);
        \draw (4) -- (5);
        \draw (5) -- (1);
    \end{tikzpicture}
\]

\subsection{Igualdad de grafos}
Se dice que dos grafos $G_1 = (V_1, E_1)$ y $G_2 = (V_2, E_2)$ son isomorfos (denotado como $G_1 \cong G_2$) si se cumple que
\[ f \text{ biyectiva}: V_1 \rightarrow V_2 \quad ; \quad \{u,v\} \in E_1 \leftrightarrow \{f(u), f(v)\} \in E_2 \]

La relación de isomorfismo $\cong$ es una relación refleja, simétrica y transitiva. Se dice que una propiedad de un grafo $G$ es preservada bajo isomorfismo, si $G$ tiene una propiedad y $G \cong G'$, entonces $G'$ también tiene la propiedad.

\subsection{Vertices y Grados}
Para un grafo $G = (V,E)$ y dos vertices $u,v \in V$
\begin{itemize}
    \item $u$ y $v$ son adyacentes si y solo si $\{u,v\} \in E$
    \item El grado de $u$ se define como: $\deg(u) = |\{ v \in V | \{u,v\} \in E \}|$
\end{itemize}

\textbf{Lema:} (handshaking)
\begin{itemize}
    \item Para todo grafo $G = (V,E)$ se cumple que $\sum_{v \in V} = 2 \cdot |E|$ (en palabras más simples, la suma de todos los grados de un grafo cualquiera siempre es $2 \cdot |E|$).
    \item Todo grafo tiene una cantidad par de vertices con grado impar.
\end{itemize}

\subsection{Subgrafos}
Un grafo $G' = (V',E')$ es un subgrafo de $G = (V,E)$ si $V' \subseteq V$ y $E' \subseteq E$, y se escribe como $G' \subseteq G$.

\subsection{Coloraciones}
Las coloraciones en un grafo permiten visualizar algunas cosas con mayor facilidad. Un claro ejemplo de esto es en un \textit{Grafo de conflictos}.\footnote{Esto queda mejor explicado con un ejemplo, el cual puede encontrar en las presentaciones de clase (Clase29 - Coloraciones, caminos y ciclos). Sería muy tedioso colocarlo acá y ya hicieron un buen trabajo en la presentación.}

Una k-coloración de un grafo $G = (V,E)$ es una función $C: V \rightarrow \{0, 1, \ldots, k-1\}$ tal que para todo $u,v \in V$, si $\{u,v\} \in E$, entonces $C(u) \not= C(v)$.

Denotaremos como $\chi(G)$ al mínimo valor que puede adoptar $k$ tal que $G = (V,E)$ tenga una k-coloración. A este valor $\chi(G)$ lo llamaremos ``número cromático''. Cabe destacar que encontrar el número cromático es un problema muy complicado (se resume en colorear manualmente el grafo y probar hasta que se logre encontrar un valor que no se pueda disminuir).

\subsection{Caminos}
Un camino de un grafo $G = (V,E)$ corresponde a una secuencia de nodos del grafo, tal que cada nodo este conectado con el siguiente de la cadena, según $E$.

\subsubsection{Camino simple}
Un camino simple de un grafo $G = (V,E)$ corresponde a lo mismo que se definió anteriormente, pero además, añadiendo el requisito de que no se repitan nodos en la secuencia (o sea, que el camino no pase por el mismo lugar más de una vez).

\subsubsection{Camino cerrado}
Un camino cerrado de un grafo $G + (V,E)$ corresponde a lo mismo que se definió anteriormente, pero este camino también debe finalizar en el mismo lugar en donde comenzó. De manera más formal\dots
\[ \text{Path} = v_0, v_1, \ldots, v_n \text{ tal que } \{ v_i, v_{i+1} \} \in E\ \forall i \in \{0, 1, \ldots, n-1\} \wedge v_0 = v_n \]

\subsubsection{Ciclo}
Un ciclo corresponde a un ciclo cerrado, en donde además, ningún nodo de la secuencia que define el camino se puede repetir.

\subsection{Conexidad}
Primero que nada, se dice que dos vertices de un grafo estan conectados si es que existe un camino de uno al otro. De ahí, sale la definición de un grafo conexo, el cual consiste en un grafo en donde todos los nodos estan conectados entre sí.

Para un grafo $G = (V,E)$, se define la relación $R_G: V \times V$, tal que $(u,v) \in R_G$ si y solo si $u$ esta conectado con $v$.

\subsection{Caminos y Tours Eulerianos}
Se dice que un camino es Euleriano si es que este camino recorre todas las aristas de un grafo, exactamente una vez. Por otro lado, un tour Euleriano corresponde a un camino Euleriano cerrado.

\subsubsection{¿Cómo verificamos si un grafo tiene un tour Euleriano?}
Sea $G = (V,E)$ un grafo conexo. $G$ tiene un tour Euleriano si y solo si, todo vértice en $G$ tiene grado par.

\subsection{Caminos y Tours Hamiltoneanos}
Se dice que un camino es Hamiltoneano si es que este camino recorre todos los nodos de un grafo, exactamente una vez. Por otro lado, un tour Hamiltoneano corresponde a un camino Hamiltoneano cerrado.



\end{document}