\documentclass[../main.tex]{subfiles}

\begin{document}

\section{Anexo}
\subsection{Ejercicio - Inferencia lógica de predicados}
\[
    \begin{tabular}{l}
        Algún estudiante en la sala no estudió para el examen.\\
        Todos los estudiantes de la sala pasaron el examen.\\ \hline
        Algún estudiante pasó el examen y no estudió.
    \end{tabular}
\]
¿Cómo modelamos este problema?
$$S(x) := x \text{ está en la sala}$$
$$E(x) := x \text{ estudió para el examen}$$
$$X(x) := x \text{ pasó el examen}$$
Entonces, la consecuencia lógica quedaría así:
\[
    \begin{tabular}{l}
        $\exists{} x . S(x) \wedge{} \neg{} E(x)$\\
        $\forall{} x . S(x) \rightarrow{} X(x)$\\ \hline
        $\exists{} x . X(x) \wedge{} \neg{} E(x)$
    \end{tabular}
\]
¿Cómo inferimos esta consecuencia lógica?
\begin{enumerate}
    \item $\exists x . S(x) \wedge \neg E(x)$ (Premisa)
    \item $S(a) \wedge \neg E(a)$ (Instanciación Existencial 1.)
    \item $S(a)$ (Simplificación Conjuntiva 2.)
    \item $\forall x . S(x) \rightarrow X(x)$ (Premisa)
    \item $S(a) \rightarrow X(a)$ (Instanciación Universal 4.)
    \item $X(a)$ (Modus ponens 3. y 5.)
    \item $\neg E(a)$ (Simplificación Conjuntiva 2.)
    \item $X(a) \wedge \neg E(a)$ (Conjunción 6. y 7.)
    \item $\exists x . X(x) \wedge \neg E(x)$ (Generalización Existencial 8.)
\end{enumerate}

\end{document}