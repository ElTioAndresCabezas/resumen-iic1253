\documentclass{article}
\usepackage{amsmath}
\usepackage{fullpage}

\title{Resumen general de Matemáticas Discretas}
\author{Andrés Cabezas y Sebastián Poblete}

\begin{document}

\maketitle

\section{¿Qué son las matemáticas discretas?}
``El lenguaje necesario para entender y modelar la computación''. Las matemáticas discretas usan conjuntos finitos e infinitos al momento de estudio. Modelan los objetos y conceptos abstractos de las matemáticas que pueden ser representados dentro de un computador.
\subsection{Lógica}
La lógica consiste en el uso y estudio del razonamiento válido. Para esto, es necesario un lenguaje, pero esto también supone un problema: Los lenguajes que los humanos hablan tiene ciertas subjetividades y diferencias entre si, lo que conduce a errores al momento de usar la lógica. Para resolver esto, es necesario usar un lenguaje formal.\\
Durante el curso, se estudiarán dos lógicas (o lenguajes), sin embargo, existen muchos más
\begin{itemize}
    \item Lógica Proposicional
    \item Lógica de Predicados
\end{itemize}
\textbf{¿Para qué son necesarias estas lógicas?} Recordemos nuestro objetivo. Queremos usar esto para realizar nuestro razonamiento matemático. De esta forma, podemos definir correctamente objetos matemáticos, teorías matemáticas y realizar demostraciones más formales

\section{Lógica Proposicional (LP)}
\subsection{Proposición}
Una proposición consiste en una afirmación, la cual puede ser \textit{verdadera (1)} o \textit{falsa (0)}.\\
Para denotar proposiciones básicas, usaremos letras mayusculas (Ej.: P, Q, R\ldots)

\subsection{Conectivos Lógicos}
La LP usa conectivos sencillos para conseguir formar proposiciones más complejas. \\
\[
    \begin{tabular}{|c|c|c|c|}
        \hline
        Conectivos & Nombre & Uso & Significado\\ \hline
        $\wedge$ & Conjunción & $P \wedge{}Q$ & P y Q \\
        $\vee$ & Disyunción & $P \vee{}Q$ & P o Q \\
        $\neg$ & Negación & $\neg{}P$ & No P \\
        $\rightarrow$ & Condicional & $P \rightarrow{}Q$ & Si P, entonces Q \\
        $\leftrightarrow$ & Bicondicional & $P \leftrightarrow{}Q$ & P, si y solo si, Q \\ \hline
    \end{tabular}
\]
\subsubsection{Conjunción ($\wedge$)}
El valor de verdad de una conjunción es \textit{verdadero} si ambas proposiciones (a cada lado del signo) son verdaderas. En cualquier otro caso, es \textit{falso}.
\[
    \begin{tabular}{cc|c}
        P & Q & $P \wedge{}Q$\\ \hline
        0 & 0 & 0\\
        0 & 1 & 0\\
        1 & 0 & 0\\
        1 & 1 & 1
    \end{tabular}
\]

\subsubsection{Disyunción ($\vee$)}
El valor de verdad de una disyunción es \textit{verdadero} si al menos una de las proposiciones (a cada lado del signo), es verdadera.
\[
    \begin{tabular}{cc|c}
        P & Q & $P \vee{}Q$\\ \hline
        0 & 0 & 0\\
        0 & 1 & 1\\
        1 & 0 & 1\\
        1 & 1 & 1
    \end{tabular}
\]

\subsubsection{Negación ($\neg$)}
El valor de verdad corresponde al opuesto del valor entregado (a la derecha del signo)
\[
    \begin{tabular}{c|c}
        P & $\neg{}P$\\ \hline
        1 & 0\\
        0 & 1 
    \end{tabular}
\]

\subsubsection{Condicional ($\rightarrow$)}
El valor de verdad de una condicional del tipo $P \rightarrow{}Q$ es \textit{falso} si P es verdadero, pero Q es falso. En cualquier otro caso, es \textit{verdadero}.\\
\[
    \begin{tabular}{cc|c}
        P & Q & $P \rightarrow{}Q$\\ \hline
        0 & 0 & 1\\
        0 & 1 & 1\\
        1 & 0 & 0\\
        1 & 1 & 1
    \end{tabular}
\]
Hint: \textit{``Si P es verdadero, entonces necesariamente Q es verdadero''}. Si P es verdadero, entonces Q deberá ser verdadero para tener un valor de verdad \textit{verdadero}. Si P es falso, entonces de forma automática el valor de verdad es \textit{verdadero}.

\subsubsection{Bicondicional ($\leftrightarrow$)}
El valor de verdad de una bicondicional es verdadero si ambas proposiciones (a ambos lados del signo) son iguales (en otras palabras, ambas verdaderas o ambas falsas).
\[
    \begin{tabular}{cc|c}
        P & Q & $P \leftrightarrow{}Q$\\ \hline
        0 & 0 & 1\\
        0 & 1 & 0\\
        1 & 0 & 0\\
        1 & 1 & 1
    \end{tabular}
\]

\subsection{Proposición Compuesta}
Una proposición es compuesta si corresponde a la negación ($\neg$), conjunción ($\wedge$), disyunción($\vee$), condicional($\rightarrow$) o bicondicional( $\leftrightarrow$ ) de proposiciones compuestas.\\
Como por ejemplo


$$P \wedge (Q \vee{} R)$$
$$\neg(P \vee (\neg{}R \wedge{} Q))$$
$$(P \rightarrow{} Q) \leftrightarrow{} (P \wedge{} Q)$$



Si se desea obtener el valor de verdad de alguna proposición compuesta, se debe evaluar de forma recursiva cada uno de los conectivos lógicos presentes.\\
Por ejemplo:

$$\neg(P \vee (\neg{} R \wedge{} Q)) \textrm{ con P = 0, Q = 1 y R = 0}$$
$$\neg(0 \vee (\neg{} 0 \wedge{} 1))$$
$$\neg(0 \vee (1 \wedge{} 1))$$
$$\neg(0 \vee 1)$$
$$\neg{}1$$
$$0$$
\\
$$(P \rightarrow{} Q) \leftrightarrow{} (P \wedge{} Q) \text{ con P = 1 y Q = 0}$$
$$(1 \rightarrow{} 0) \leftrightarrow{} (1 \wedge{} 0)$$
$$0 \leftrightarrow{} 0$$
$$1$$

\subsubsection{Paréntesis y prioridad}
El orden de prioridad entre conectivos lógicos, al momento de evaluar proposiciones compuestas, será el siguiente:
\[
    \begin{tabular}{c|c}
        Conectivo & Precedencia\\ \hline
        $\neg$ & 1\\
        $\wedge$ & 2\\
        $\vee$ & 3\\
        $\rightarrow$ & 4\\
        $\leftrightarrow$ & 5
    \end{tabular}
\]

\section{Formulas y Valuaciones}
\subsection{Variables Proposicionales}
Una variable proposicional es una variable que puede ser reemplazada con los valores 1 o 0. Generalmente son representadas con una letra minúscula (Amiga eri boolean)

\subsection{Formulas Proposicionales}
Una formula proposicional es una formula que puede ser
\begin{itemize}
    \item Una variable proposicional
    \item Los valores 1 o 0
    \item Una combinación con conectivos lógicos
\end{itemize}
Generalmente son representadas con letras griegas (Ej.: $\alpha$)\\
Ejemplos:
$$\alpha{}(p,q,r) := p \wedge{}(q \rightarrow{} r)$$
$$\beta{}(p,q) := (p \wedge{} \neg{}q) \vee{} (\neg{}p \wedge{} 1)$$

\section{Equivalencia Lógica}
\subsection{Definición}
Si tenemos dos formulas proposicionales con las mismas variables proposicionales
$$\alpha{}(p_{1},\ldots{},p_{n}) \text{ y } \beta{}(p_{1},\ldots{},p_{n})$$
Entonces, $\alpha$ y $\beta$ serán logicamente equivalentes
$$\alpha \equiv \beta$$
si para toda valuacion posible $(v_{1}, \ldots{},v_{n})$ se cumple que:
$$\alpha{}(v_{1}, \ldots{},v_{n}) = \beta{}(v_{1}, \ldots{},v_{n})$$
Ejemplo:
Para las fórmulas $p \wedge (q \vee r)$ y $(p \wedge q) \vee (p \wedge r)$ se tiene la siguiente tabla de verdad:
\[
    \begin{tabular}{ccc|cc}
        p & q & r & $p \wedge (q \vee r)$ & $(p \wedge q) \vee (p \wedge r)$ \\ \hline
        0 & 0 & 0 & 0 & 0\\
        0 & 0 & 1 & 0 & 0\\
        0 & 1 & 0 & 0 & 0\\
        0 & 1 & 1 & 0 & 0\\
        1 & 0 & 0 & 0 & 0\\
        1 & 0 & 1 & 1 & 1\\
        1 & 1 & 0 & 1 & 1\\
        1 & 1 & 1 & 1 & 1
    \end{tabular}
\]
Como ambas formulas son equivalentes para toda valuación, entonces:
$$p \wedge (q \vee r) \equiv (p \wedge q) \wedge (p \wedge r)$$

\subsection{Equivalencias útiles}
\begin{enumerate}
    \item Conmutatividad: 
                        $$p \wedge q \equiv q \wedge p$$ 
                        $$p \vee q \equiv q \vee p$$
    \item Asociatividad: 
                        $$p \wedge (q \wedge r) \equiv (p \wedge q) \wedge r$$
                        $$p \vee (q \vee r) \equiv (p \vee q) \vee r$$
    \item Idempotente:
                        $$p \wedge p \equiv p$$
                        $$p \vee p \equiv p$$
    \item Doble negación:
                        $$\neg\neg p \equiv p$$
    \item Distributividad:
                        $$p \wedge (q \vee r) \equiv (p \wedge q) \vee (p \wedge r)$$
                        $$p \vee (q \wedge r) \equiv (p \vee q) \wedge (p \vee r)$$
    \item De Morgan:
                        $$\neg (p \wedge q) \equiv \neg p \vee \neg q$$
    \item Implicación:
                        $$p \rightarrow q \equiv \neg p \vee q$$
                        $$p \rightarrow q \equiv \neg q \rightarrow \neg p$$
                        $$p \leftrightarrow q \equiv (p \rightarrow q) \wedge (q \rightarrow p)$$
    \item Absorción:
                        $$p \vee (p \wedge q) \equiv p$$
                        $$p \wedge (p \vee q) \equiv p$$
    \item Identidad:
                        $$p \vee 0 \equiv p$$
                        $$p \wedge 1 \equiv p$$
    \item Dominación:
                        $$p \wedge 0 \equiv 0$$
                        $$p \vee 1 \equiv 1$$
\end{enumerate}
\end{document}