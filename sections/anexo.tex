\documentclass[../main.tex]{subfiles}

\begin{document}

\section{Anexo}
\subsection{Ejercicio - Inferencia lógica de predicados}
\[
    \begin{tabular}{l}
        Algún estudiante en la sala no estudió para el examen.\\
        Todos los estudiantes de la sala pasaron el examen.\\ \hline
        Algún estudiante pasó el examen y no estudió.
    \end{tabular}
\]
¿Cómo modelamos este problema?
$$S(x) := x \text{ está en la sala}$$
$$E(x) := x \text{ estudió para el examen}$$
$$X(x) := x \text{ pasó el examen}$$
Entonces, la consecuencia lógica quedaría así:
\[
    \begin{tabular}{l}
        $\exists{} x . S(x) \wedge{} \neg{} E(x)$\\
        $\forall{} x . S(x) \rightarrow{} X(x)$\\ \hline
        $\exists{} x . X(x) \wedge{} \neg{} E(x)$
    \end{tabular}
\]
¿Cómo inferimos esta consecuencia lógica?
\begin{enumerate}
    \item $\exists x . S(x) \wedge \neg E(x)$ (Premisa)
    \item $S(a) \wedge \neg E(a)$ (Instanciación Existencial 1.)
    \item $S(a)$ (Simplificación Conjuntiva 2.)
    \item $\forall x . S(x) \rightarrow X(x)$ (Premisa)
    \item $S(a) \rightarrow X(a)$ (Instanciación Universal 4.)
    \item $X(a)$ (Modus ponens 3. y 5.)
    \item $\neg E(a)$ (Simplificación Conjuntiva 2.)
    \item $X(a) \wedge \neg E(a)$ (Conjunción 6. y 7.)
    \item $\exists x . X(x) \wedge \neg E(x)$ (Generalización Existencial 8.)
\end{enumerate}

\subsection{Demostración - Mínimo único}
\label{sec:dem_min_unico}
\textbf{Si $S$ tiene un elemento mínimo, entonces dicho elemento es único.}\\
Sea $x_1^{\downarrow} \in S$ y $x_2^{\downarrow} \in S$ ambos mínimos y $x_1 \not= x_2$. ¿Es esto posible?\\
Mediante la definición de mínimo, llegamos a que
\[ \forall y \in S . x_1^{\downarrow} \preceq y \quad ; \quad \forall y \in S . x_2^{\downarrow} \preceq y \]
Esto nos lleva a que
\[
    \left.
        \begin{array}{c}
            x_1^{\downarrow} \preceq x_2^{\downarrow}\\
            x_2^{\downarrow} \preceq x_1^{\downarrow}
        \end{array}
    \right \}
    x_1^{\downarrow} = x_2^{\downarrow}
\]
Así, queda demostrado que el mínimo debe ser único.

\subsection{Demostración - Si es mínimo, es minimal}
\label{sec:dem_min_minimal}
\textbf{Si $x$ es mínimo, entonces $x$ es minimal}\\
Sea $x^{\downarrow}$ un mínimo\\
Para realizar la demostración, se debe demostrar que $\forall z \in S . z \preceq x^{\downarrow} \rightarrow z = x$\\
Suponga que $z \preceq x^{\downarrow}$. Por definición de mínimo, se tiene que $x^{\downarrow} \preceq z$. Entonces, considerando toda la información, se llega a que
\[
    \left.
        \begin{array}{c}
            z \preceq x^{\downarrow}\\
            x^{\downarrow} \preceq z
        \end{array}
    \right \}
    z = x^{\downarrow}
\]

\subsection{Demostración - Teorema de Cantor}
Por demostrar: No existe biyección entre $A$ y $2^A = \{ S | S \subseteq A \}$\\
Para ayudar a que sea más sencillo entender la demostración, se puede demostrar primero que no existe biyección entre $\mathds{N}$ y $2^{\mathds{N}}$, para después extrapolarlo a cualquier conjunto $A$.\\
Por demostrar: No existe biyección entre $\mathds{N}$ y $2^{\mathds{N}} = \{ S | S \subseteq \mathds{N} \}$\\
Para comenzar la demostración, se supone que si existe una biyección $f: \mathds{N} \rightarrow 2^{\mathds{N}}$
\[
    \begin{tabular}{c|cccccc}
        \text{} & $0$ & $1$ & $2$ & $3$ & $4$ & $\ldots$\\\hline
        $f(0)$ & $1$ & $1$ & $0$ & $1$ & $0$ & $\ldots$\\
        $f(1)$ & $0$ & $0$ & $1$ & $1$ & $1$ & $\ldots$\\
        $f(2)$ & $1$ & $1$ & $1$ & $1$ & $0$ & $\ldots$\\
        $f(3)$ & $1$ & $0$ & $1$ & $0$ & $0$ & $\ldots$\\
        $f(4)$ & $0$ & $0$ & $1$ & $1$ & $0$ & $\ldots$\\
        $\dots$ & & & $\ldots$ & & &
    \end{tabular}
\]
Aquí, la coordenada $(i,j)$ es $1$ si y solo si $j \in f(i)$.
Si consideramos todos los elementos de la diagonal y los colocamos en un conjunto $D$\dots
\[
    \begin{tabular}{c|cccccc}
        \text{} & $0$ & $1$ & $2$ & $3$ & $4$ & $\ldots$\\\hline
        $f(0)$ & \cellcolor{lightgray}$1$ & $1$ & $0$ & $1$ & $0$ & $\ldots$\\
        $f(1)$ & $0$ & \cellcolor{lightgray}$0$ & $1$ & $1$ & $1$ & $\ldots$\\
        $f(2)$ & $1$ & $1$ & \cellcolor{lightgray}$1$ & $1$ & $0$ & $\ldots$\\
        $f(3)$ & $1$ & $0$ & $1$ & \cellcolor{lightgray}$0$ & $0$ & $\ldots$\\
        $f(4)$ & $0$ & $0$ & $1$ & $1$ & \cellcolor{lightgray}$0$ & $\ldots$\\
        $\dots$ & & & $\ldots$ & & &
    \end{tabular}
\]
\[ D = \{ i \in \mathds{N} | i \in f(i) \} \in 2^{\mathds{N}} = \{ 0, 2, \ldots \} \]
Si pensamos en el complemento de $D$, llamese, el conjunto de elementos que no se encuentran contenidos en el conjunto $D$, obtenemos
\[ \bar{D} = \{ i \in \mathds{N} | i \not\in f(i) \} \in 2^{\mathds{N}} = \{ 1, 3, 4, \ldots \} \]
Por la naturaleza de $\bar{D}$, no hay ninguna $f(x)$ para todo $x$, debido a que para que $x \in f(x)$, se debe cumplir que $x \not\in \bar{D}$. Con esto se llega a que no existe una biyección entre $\mathds{N}$ y $2^{\mathds{N}}$.\\
Este mismo argumento se puede extrapolar a cualquier conjunto $A$, ya que el motivo por el que ocurre esto no esta relacionado con el conjunto en si, si no por la forma en la que funciona el conjunto potencia.\\
Si se mantiene escéptico, considere
\[ \bar{D} = \{ a \in A | a \not\in f(a) \} \]
Ahora considere un $x^* \in A$, tal que $f(x^*) = \bar{D}$
\begin{itemize}
    \item $x^* \in f(x^*) \Rightarrow x^* \in \bar{D} \Rightarrow x^* \not\in f(x^*)$
    \item $x^* \not\in f(x^*) \Rightarrow x^* \in \bar{D} \Rightarrow x^* \in f(x^*)$
\end{itemize}
Llegamos a una contradicción, lo que significa que dicho $x^*$ no puede existir, llegando a que no existe una biyección.


\end{document}