\documentclass[../main.tex]{subfiles}

\begin{document}
\section{Teoría de Números}
\subsection{División}
Considerando a $\mathds{Z}$ como el conjunto de los números enteros, para $a, b \in \mathds{Z}$ y $a \not= 0$, se dice que $a$ divide a $b$ si es que:
\[ a | b \leftrightarrow \exists q \in \mathds{Z}\ .\ a \cdot q = b \]
En el caso que esto no se cumpla, entonces $a$ no divide $b$, expresado como $a \slashed{|} b$
\subsubsection{Propiedades de la división}
\begin{itemize}
    \item Si $a|b$ y $a|c$, entonces $a|(b + c)$
    \item Si $a|b$, entonces $a|(b \cdot c)$ para todo $c \in \mathds{Z}$
    \item Si $a|b$ y $b|c$, entonces $a|c$
\end{itemize}
\textbf{Corolario:} Si $a|b$ y $b|c$, entonces $a|(b \cdot m + n \cdot c)$, para todo $m, n \in \mathds{Z}$

\subsection{Módulo}
Con $a, b \in \mathds{Z}$, $a > 0$ y $a|b$, entonces existe un único par $q,r \in \mathds{Z}$ tal que
$a \cdot q + r = b$. Esta corresponde a la definición de la división con resto. Mediante esta, se define el operador módulo (\modulo) y el operador división (\division).
\[
    \begin{tabular}{rcl}
        $b \division a$ & $=$ & $q$\\
        $b \modulo a$ & $=$ & $r$
    \end{tabular}
\]

\subsection{Congruencia modular}
Con $m \in \mathds{Z}$ y $m > 0$, diremos que para todo $a,b \in \mathds{Z}$, $a$ es congruente con $b \modulo m$ si:
\[ a \equiv b (\modulo m) \quad \text{si, y solo si} \quad m | (a - b) \]

\subsubsection{Propiedades de la congruencia modular}
Para todo $a,b,m \in \mathds{Z}$, donde $m > 0$, se cumple que:
\begin{itemize}
    \item $a \equiv b (\modulo m)$
    \item $a = b + m \cdot s$, para algún $s \in \mathds{Z}$
    \item $(a \modulo m) = (b \modulo m)$
\end{itemize}

\subsubsection{Suma y multiplicación de la congruencia modular}
Para todo $m > 0$, si $a \equiv b (\modulo m)$ y $c \equiv d (\modulo m)$ entonces:
\[
    \begin{tabular}{rcl}
        $a + c$ & $\equiv$ & $b + d (\modulo m)$\\
        $a \cdot c$ & $\equiv$ & $b \cdot d (\modulo m)$
    \end{tabular}
\]
Adicionalmente\dots
\[
    \begin{tabular}{rcl}
        $(a + b) \modulo m$ & $=$ & $((a \modulo m) + (b \modulo m)) \modulo m$\\
        $(a \cdot b) \modulo m$ & $=$ & $((a \modulo m) \cdot (b \modulo m)) \modulo m$
    \end{tabular}
\]

\subsubsection[Aritmética módulo m]{Aritmética módulo $m$ - Aritmética modular}
Con $m > 0$, se define $\mathds{Z}_m = \{ 0, 1, 2, \ldots, m - 1 \}$. Entonces, para todo $a,b \in \mathds{Z}_m$, se definen las operaciones $+_m$ y $\cdot_m$
\[
    \begin{tabular}{rcl}
        $a +_m b$ & $=$ & $(a+b) \modulo m$\\
        $a \cdot_m b$ & $=$ & $(a \cdot b) \modulo m$
    \end{tabular}
\]

La aritmética modular cumple con las siguientes propiedades:
\[
    \begin{tabular}{rl}
        Clausura: & $a +_m b \in \mathds{Z}_m \quad ; \quad a \cdot_m b \in \mathds{Z}_m$\\
        Conmutatividad & $a +_m b = b +_m a \quad ; \quad a \cdot_m b = b \cdot_m a$\\
        Asociatividad & $a +_m (b +_m c) = (a +_m b) +_m c \quad ; \quad a \cdot_m (b \cdot_m c) = (a \cdot_m b) \cdot_m c$\\
        Identidad: & $a +_m 0 = a \quad ; \quad a \cdot_m 1 = a$\\
        Inverso aditivo: & $a \not= 0,\ \exists a' \in \mathds{Z}_m\ .\ a +_m a' = 0$\\
        Distributividad: & $a \cdot_m (b +_m c) = (a \cdot_m b) + (a \cdot_m c)$\\
    \end{tabular}
\]

\subsection{Representación de los números}
Sea $b > 1$. Si $n \in \mathds{N} - \{ 0 \}$, entonces $n$ se puede escribir de forma única como:
\[ n = a_{k-1} b^{k-1} + a_{k-2} b^{k-2} + a_{k-3} b^{k-3} + \ldots + a_{2} b^{2} + a_{1} b^{1} + a_{0} = \sum_{i = 0}^{k - 1} a_i b^i \]
con
\begin{itemize}
    \item $k \geq 1$
    \item Para todo $i < k$, $a_i < b$
    \item $a_{k-1} \not= 0$
\end{itemize}

Para poder simplificar un poco las cosas, vamos a establecer que todo número en representación de n en base b corresponde a la secuencia
\[ (n)_b = a_{k-1} \ldots a_2 a_1 a_0 \]

Como pequeño dato curioso, si esto le parece familiar al lector, es porque esto representa la forma en que nosotros escribimos los números normalmente, en donde la base $b = 10$.

\subsubsection[Encontrando la representación de n en base b]{Encontrando la representación de $n$ en base $b$}
Ahora que entendemos que podemos representar números usando distintas bases, ¿cómo podemos encontrar la representación de cualquier número en cualquier base?

Si se tiene un número $n \in \mathds{N} - \{ 0 \}$ y $b > 0$, sabiendo que su representación debe ser de la forma $(n)_b = a_{k-1} \ldots a_2 a_1 a_0 $ y que por la división con resto, sabemos que $n = q \cdot b + r$, entonces tenemos que
\[
    \begin{tabular}{rcl}
        $r$ & $=$ & $a_0$\\
        $(q)_b$ & $=$ & $a_{k-1} \ldots a_1$\\
    \end{tabular}
\]

\subsubsection[Suma de números en base b]{Suma de números en base $b$}
La forma de llegar al algoritmo para la suma de números en base $b$ corresponde al siguiente, considerando a que $n$ y $m$ son números en base $b$.
\[
    \begin{tabular}{rcll}
        $n + m$ & $=$ & $(n_{k-1} + m_{k-1}) \cdot b^{k-1} + \ldots + (n_{2} + m_{2}) \cdot b^{2} + (n_{1} + m_{1}) \cdot b + (n_{0} + m_{0})$ & /$(n_{0} + m_{0}) = c_0 \cdot b + s_0$\\
        $n + m$ & $=$ & $(n_{k-1} + m_{k-1}) \cdot b^{k-1} + \ldots + (n_{2} + m_{2}) \cdot b^{2} + (n_{1} + m_{1} + c_0) \cdot b + s_0$ & /$(n_{1} + m_{1} + c_0) = c_1 \cdot b + s_1$\\
        $n + m$ & $=$ & $(n_{k-1} + m_{k-1}) \cdot b^{k-1} + \ldots + (n_{2} + m_{2} + c_1) \cdot b^{2} + s_1 \cdot b + s_0$ & /$(n_{2} + m_{2} + c_1) = c_2 \cdot b + s_2$\\
        &&&/ \ldots
    \end{tabular}
\]

Si se continua aplicando hasta terminar con toda la ecuación, se obtiene que
\[ n + m = c_{k-1} \cdot b^k + s_{k-1} \cdot b^{k-1} + \ldots + s_{1} \cdot b + s_0 \]

Estas son muchas letras y posiblemente confunde demasiado, por lo que es mejor trabajar con un ejemplo de como se usa este algoritmo. Supongamos que queremos realizar la suma $(11)_2 + (14)_2$, donde $(11)_2 = 1011$ y $(14)_2 = 1110$.
\[
    \begin{tabular}{rrcll}
        Se comienza con el primer dígito (101\textbf{1}; 111\textbf{0}): & $1 + 0$ & $=$ & $0 \cdot 2 + 1$ & Resultado: $1$\\
        Se continua con el segundo dígito (10\textbf{1}1; 11\textbf{1}0): & $1 + 1 + 0$ & $=$ & $1 \cdot 2 + 0$ & Resultado: $01$\\
        Se continua con el tercer dígito (1\textbf{0}11; 1\textbf{1}10): & $0 + 1 + 1$ & $=$ & $1 \cdot 2 + 0$ & Resultado: $001$\\
        Se continua con el cuarto dígito (\textbf{1}011; \textbf{1}110): & $1 + 1 + 1$ & $=$ & $1 \cdot 2 + 1$ & Resultado: $1001$\\
        Siguiente dígito de la base (5° dígito acumulado): & $0 + 0 + 1$ & $=$ & $0 \cdot 2 + 1$ & Resultado: $11001$
    \end{tabular}
\]

Entonces, $(11)_2 + (14)_2 = 11001$.

Este algoritmo es exactamente lo que se usa para realizar sumas de forma manual en base $10$.

\subsubsection[Multiplicación de números en base b]{Multiplicación de números en base $b$}


\end{document}